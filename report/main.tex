\documentclass[journal,onecolumn]{IEEEtran}

% include a few useful packages; you will probably want more
\usepackage{graphicx} 
\usepackage[margin=1in]{geometry} 
\usepackage{amsmath,amsthm,amssymb}
\usepackage[noend]{algpseudocode}
\usepackage[hypcap=true]{caption}
\usepackage{lipsum}
\usepackage{float}
\makeatletter
\def\BState{\State\hskip-\ALG@thistlm}
\makeatother
\usepackage{algorithm}
% place all photos and diagrams in the media folder
\graphicspath{{media/}}

\hyphenation{op-tical net-works semi-conduc-tor}

\begin{document}

\title{%
  ROB 550 BotLab Report \\
  %\large Prototype Implementations of SLAM and A* for a Wheeled Robot
  }
    
\author{Saptadeep Debnath - saptadeb@umich.edu}

\maketitle

\IEEEpeerreviewmaketitle

\section{MAPPING AND SLAM}

\subsection*{1.1 - Mapping - Occupancy Grid} 

Write a paragraph or two description including the following:

\begin{itemize}
    \item Provide the values of the incremental log odds you're using for a free \& occupied cells.
    \item Describe the method used determine which cells to update.
    \item Comment on mapping behaviour.
\end{itemize}



\subsection*{1.2.1 - MCL - Action Model} 

 Write a paragraph or two description including the following:
 \begin{itemize}
    \item Specify what type of action model you're using.
    \item Provide all noise constants that you're using.
    \item Comment on the your action model:
        \begin{itemize}
            \item Does the distribution grow as expected for the particles?
            \item Do you think the action model constants will differ drastically from one
mbot to another?
        \end{itemize}
\end{itemize}

\begin{figure}[H]
\centering
\includegraphics[width=0.3\textwidth]{Media/1211.png}
\caption{Final png showing the particle distribution you obtained at the end of drive square with action only}
\end{figure}

\begin{figure}[H]
\centering
\includegraphics[width=0.3\textwidth]{Media/1212.png}
\caption{Final png showing the particle distribution you obtained at the end of obstacle slam with action only}
\end{figure}

\begin{figure}[H]
\centering
\includegraphics[width=0.3\textwidth]{Media/1213.png}
\caption{Final png showing the particle distribution you obtained at the end of straight line calm with action only}
\end{figure}

\subsection*{1.2.2 - MCL - Sensor Model \& Particle Filter} 

 Write a paragraph or two description including the following:
 \begin{itemize}
    \item Specify what type of sensor model you're using, and how you are weighting and resampling.
    \item Provide all noise constants that you're using.
    \item Is the estimated pose closer to the true pose with your localized pose estimate compared to that from odometry?
    \item Do the particles remain in a tight region as the robot moves?
    \item Do the particles spread more aggressively with a certain motion type? (rotation, for example).
\end{itemize}

\begin{figure}[H]
\centering
\includegraphics[width=0.3\textwidth]{Media/1221.png}
\caption{Final png showing the particle distribution you obtained at the end of drive square with localization only}
\end{figure}

\begin{figure}[H]
\centering
\includegraphics[width=0.3\textwidth]{Media/1222.png}
\caption{Final png showing the particle distribution you obtained at the end of obstacle slam with localization only}
\end{figure}


\begin{figure}[H]
\centering
\includegraphics[width=0.3\textwidth]{Media/1223.png}
\caption{Final png showing the particle distribution you obtained at the end of straight line with localization only}
\end{figure}

\subsection*{1.3 - Simultaneous Localization and Mapping (SLAM)} 

 Write a paragraph or two description including the following:
 \begin{itemize}
    \item Did you obtain similar/close performance for obstacle\_slam\_10mx10m\_5cm.log compared to the previous two runs? Why or why not?
    \item Did the change of map resolution affect your computation time? Did it improve/harm your slam results?
    \item Did you have to change your mapping log odds to improve on performance? If so, report them.
\end{itemize}

\begin{figure}[H]
\centering
\includegraphics[width=0.3\textwidth]{Media/131.png}
\caption{Final png showing the particle distribution you obtained at the end of Obstacle Slam with Full SLAM}
\end{figure}

\begin{figure}[H]
\centering
\includegraphics[width=0.3\textwidth]{Media/template-robotics.jpg}
\caption{Final png showing the particle distribution you obtained at the end of Maze LowRes with Full SLAM}
\end{figure}

\begin{figure}[H]
\centering
\includegraphics[width=0.3\textwidth]{Media/template-robotics.jpg}
\caption{Final png showing the particle distribution you obtained at the end of Maze HiRes with Full SLAM}
\end{figure}

\begin{figure}[H]
\centering
\includegraphics[width=0.3\textwidth]{Media/template-robotics.jpg}
\caption{Final png showing the particle distribution you obtained at the end of Maze Hard with Full SLAM}
\end{figure}

\section{PATH PLANNING AND EXPLORATION}

\subsection*{2.1 - Obstacle Distances} 

Comment on whether your code passed or failed the three tests in obstacle\_distance\_grid\_test. Provide remarks on the code’s performance and whether there is anything major slowing it down.

\begin{figure}[H]
\centering
\includegraphics[width=0.3\textwidth]{Media/21.png}
\caption{Final png showing the obstacle distance grip obtained at the end of Obstacle Slam}
\end{figure}


\subsection*{2.2 - A* Path Planning} 

Report statistics on your path planning execution times for each of the example problems in the data/astar folder. If your algorithm is optimal and fast, great. If not please discuss possible reasons and strategies for improvement.

\begin{figure}[H]
\centering
\includegraphics[width=0.3\textwidth]{Media/template-robotics.jpg}
\caption{Final png showing the path you obtained at the end of Maze HiRes with A* path planning}
\end{figure}

\subsection*{2.3 - Map Exploration} 

Comment on your exploration performance:
 \begin{itemize}
    \item What are the factors that are preventing your exploration from being ideal?
    \item Provide your logic behind tackling frontiers (how do you go about chosing your next frontier to go to)
    \item Comments on the performance of Astar and path planning with automated exploration commands.
\end{itemize}

\ifCLASSOPTIONcaptionsoff
  \newpage
\fi

% Include your citations in the references.bib file
\nocite{*}
\bibliographystyle{IEEEtran}
\bibliography{references.bib}
\end{document}